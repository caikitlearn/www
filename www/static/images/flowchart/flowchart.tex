\documentclass{article}

\usepackage[latin1]{inputenc}
\usepackage{tikz}
\usetikzlibrary{shapes,arrows}
\renewcommand\baselinestretch{0.75}
\fontfamily{qcr}
%%%<
\usepackage{verbatim}
\usepackage[active,tightpage]{preview}
\PreviewEnvironment{tikzpicture}
\setlength\PreviewBorder{5pt}%
%%%>

\begin{comment}
:Title: Simple flow chart
:Tags: Diagrams

With PGF/TikZ you can draw flow charts with relative ease. This flow chart from [1]_
outlines an algorithm for identifying the parameters of an autonomous underwater vehicle model.

Note that relative node
placement has been used to avoid placing nodes explicitly. This feature was
introduced in PGF/TikZ >= 1.09.

.. [1] Bossley, K.; Brown, M. & Harris, C. Neurofuzzy identification of an autonomous underwater vehicle `International Journal of Systems Science`, 1999, 30, 901-913


\end{comment}


\begin{document}
\pagestyle{empty}


% Define block styles
\tikzstyle{decision} = [diamond, draw, fill=blue!20,
    text width=4.5em, text badly centered, node distance=3cm, inner sep=0pt]
\tikzstyle{block} = [rectangle, draw, fill=blue!20,
    text width=6em, text centered, rounded corners, minimum height=3em, minimum width=7em]
\tikzstyle{small-block} = [rectangle, draw, fill=blue!20,
    text width=5em, text centered, rounded corners, minimum height=2em]
\tikzstyle{line} = [draw, -latex']
\tikzstyle{cloud} = [draw, ellipse,fill=red!20, node distance=3cm,
    minimum height=2em]

\tiny
\begin{tikzpicture}[node distance = 2cm, auto]
    % Place nodes
    \node [block] (init) {all tweets from 2009-2019 (36,241)};
    \node [block, left of=init] (gpt2data) {tweets with 100+ favs (8,013)};
    \node [block, below of=gpt2data] (gpt2model) {modified GPT-2};
    \node [block, below of=gpt2model] (gpt2output) {1,000 generated tweets};
    \node [block, right of=init] (bleu) {creating BLEU references};
    \node [block, below of=init] (scoringdata) {tweets with 5+ words (29,046)};
    \node [block, below of=scoringdata] (scoringmodel) {scoring model};
    \node [block, below of=scoringmodel] (top100) {top 100 generated tweets};
    \node [block, below of=top100] (unique) {top unique tweets};
    \node [block, below of=unique] (twitter) {publish to Twitter};
    % \node [block, left of=evaluate, node distance=3cm] (update) {update model};
    % \node [decision, below of=evaluate] (decide) {is best candidate better?};
    % \node [block, below of=decide, node distance=3cm] (stop) {stop};
    % Draw edges
    \path [line] (init) -- (bleu);
    \path [line] (gpt2data) -- node {training} (gpt2model);
    \path [line] (gpt2model) -- node {predict} (gpt2output);
    \path [line] (gpt2output) -- (scoringmodel);
    \path [line] (scoringmodel) -- node {predict} (top100);
    % \path [line] (decide) -| node [near start] {yes} (update);
    % \path [line] (decide) -- node {no}(stop);
    \path [line] (init) -- (gpt2data);
    \path [line] (scoringdata) -- node {training} (scoringmodel);
    \path [line] (init) -- (scoringdata);
    \path [line] (bleu) |- (top100);
    \path [line] (top100) -- node {removing overfitted Tweets} (unique);
    \path [line] (unique) -- node {manual filtering} (twitter);
\end{tikzpicture}

\end{document}
